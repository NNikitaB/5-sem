\documentclass[12pt,a4paper]{article}


\usepackage[T2A]{fontenc}
\usepackage[utf8]{inputenc}
\usepackage[english,russian]{babel}
\usepackage{indentfirst}
\usepackage{misccorr}
\usepackage{graphicx}
\usepackage{amsmath}
\usepackage{xcolor}
\usepackage{subcaption}
\usepackage{hyperref}
\begin{document}
	
	
	
	\begin{center}
	\large
	\section*{ \textbf{Синтез и анализ многовыходных комбинационных схем	в базисе И-ИЛИ-НЕ с учетом неопределенностей} }
	\end{center}


Дана система частично определенных булевых функций,наборы где функции определены ,представленны в табл. \ref{h1}


\begin{table}[h!]
	\caption{\label{h1} }
	\begin{center}
		\begin{tabular}{|c|c|c|c||c|c|c|}
			\hline
$x_1$ & $x_2$ & $x_3$ & $x_4$ & $f_1$ & $f_2$ & $f_3$ \\
\hline		

$0$ & $0$ & $0$ & $0$ & $0$ & $1$ & $1$ \\
\hline
$0$ & $0$ & $0$ & $1$ & $0$ & $1$ & $0$ \\
\hline
$0$ & $1$ & $0$ & $0$ & $0$ & $0$ & $1$ \\
\hline
$0$ & $1$ & $0$ & $1$ & $0$ & $0$ & $1$ \\
\hline
$0$ & $1$ & $1$ & $0$ & $1$ & $0$ & $0$ \\
\hline
$1$ & $0$ & $0$ & $0$ & $0$ & $1$ & $1$ \\
\hline
$1$ & $0$ & $0$ & $1$ & $0$ & $1$ & $1$ \\
\hline
$1$ & $0$ & $1$ & $0$ & $1$ & $1$ & $0$ \\
\hline
$1$ & $0$ & $1$ & $1$ & $1$ & $1$ & $0$ \\
\hline
$1$ & $1$ & $0$ & $1$ & $1$ & $0$ & $1$ \\
\hline
$1$ & $1$ & $1$ & $0$ & $1$ & $0$ & $1$ \\
\hline
$1$ & $1$ & $1$ & $1$ & $0$ & $0$ & $0$ \\
\hline
$0$ & $0$ & $1$ & $0$ & $1$ & $1$ & $0$ \\
\hline
$0$ & $0$ & $1$ & $1$ & $1$ & $1$ & $0$ \\
\hline
$0$ & $1$ & $1$ & $1$ & $1$ & $0$ & $0$ \\
\hline
		\end{tabular}
	\end{center}
\end{table} 


Минимизацию системы частично определенных булевых функций
выполним в три этапа:
\begin{enumerate} 
\item нахождение простых импликант системы частично определенных
булевых функций;
\item нахождение минимального покрытия импликантной матрицы
Квайна системы частично определенных булевых функций;
\item формирование минимальной системы булевых функций
\end{enumerate}


ЭТАП 1. Простой импликантой системы частично определенных булевых функций $F$ называется конъюнкция $К$, если:
\begin{enumerate} 
\item  она накрывает хотя бы один набор $a$, на котором $f_i(a) = 1$;
\item  не накрывает ни одного набора $a$, на котором $f_i(a) = 0$;
\item  любая конъюнкция, полученная из нее вычеркиванием переменных, накрывает некоторый набор $a$, на котором $f_i(a) = 1$.
\end{enumerate} 

Здесь $i \in {1, 2, …, n}$, где $n$ — количество функций в системе.
Для нахождения всех простых импликант удобно пользоваться представлением системы частично определенных булевых функций в виде
таблиц \ref{T1} и \ref{T0} , в первой из которых перечислены наборы
аргументов, на которых хотя бы одна из функций системы принимает
единичное значение, а во второй — наборы аргументов, на которых хотя
бы одна из функций системы принимает нулевое значение. Наборы таблицы \ref{T1} отмечены номерами функций, которые на этих наборах принимают единичное значение. Наборы таблицы \ref{T0} отмечены номерами
функций, которые на этих наборах принимают нулевое значение.



\begin{table}[h!]
\begin{tabular}{c c}

	%\caption{\label{T1} Наборы ,где  $f = 1$ }
%	Таблица \label{T1} & Таблица \label{T0} \\
	 Наборы ,где  $f = 1$ &  Наборы ,где  $f = 0$ \\
		\begin{tabular}{|c|c|c|c||c|}
			
			\hline
			$x_1$ & $x_2$ & $x_3$ & $x_4$ & признаки  \\
			\hline		
			
			$0$ & $0$ & $0$ & $0$ & $(2,3)$  \\
			\hline
			$0$ & $0$ & $0$ & $1$ & $(2)$  \\
			\hline
			$0$ & $1$ & $0$ & $0$ & $(3)$  \\
			\hline
			$0$ & $1$ & $0$ & $1$ & $(3)$  \\
			\hline
			$0$ & $1$ & $1$ & $0$ & $(1)$  \\
			\hline
			$1$ & $0$ & $0$ & $0$ & $(2,3)$  \\
			\hline
			$1$ & $0$ & $0$ & $1$ & $(2,3)$  \\
			\hline
			$1$ & $0$ & $1$ & $0$ & $(1,2)$ \\
			\hline
			$1$ & $0$ & $1$ & $1$ & $(1,2)$  \\
			\hline
			$1$ & $1$ & $0$ & $1$ & $(1,3)$ \\
			\hline
			$1$ & $1$ & $1$ & $0$ & $(1,3)$  \\
			\hline
			$0$ & $0$ & $1$ & $0$ & $(1,2)$  \\
			\hline
			$0$ & $0$ & $1$ & $1$ & $(1,2)$  \\
			\hline
			$0$ & $1$ & $1$ & $1$ & $(1)$  \\
			\hline
		\end{tabular}

&
%	\caption{\label{T0} Наборы ,где  $f = 0$ }

		\begin{tabular}{|c|c|c|c||c|}
%			\caption{\label{T0} Наборы ,где  $f = 0$ }
			\hline
			$x_1$ & $x_2$ & $x_3$ & $x_4$ & признаки  \\
			\hline		
			
			$0$ & $0$ & $0$ & $0$ & $(1)$  \\
			\hline
			$0$ & $0$ & $0$ & $1$ & $(1,3)$  \\
			\hline
			$0$ & $1$ & $0$ & $0$ & $(1,2)$  \\
			\hline
			$0$ & $1$ & $0$ & $1$ & $(1,2)$  \\
			\hline
			$0$ & $1$ & $1$ & $0$ & $(2,3)$  \\
			\hline
			$1$ & $0$ & $0$ & $0$ & $(1)$  \\
			\hline
			$1$ & $0$ & $0$ & $1$ & $(1)$  \\
			\hline
			$1$ & $0$ & $1$ & $0$ & $(3)$ \\
			\hline
			$1$ & $0$ & $1$ & $1$ & $(3)$  \\
			\hline
			$1$ & $1$ & $0$ & $1$ & $(2)$ \\
			\hline
			$1$ & $1$ & $1$ & $0$ & $(2)$  \\
			\hline
			$1$ & $1$ & $1$ & $1$ & $(1,2,3)$  \\
			\hline
			$0$ & $0$ & $1$ & $0$ & $(3)$  \\
			\hline
			$0$ & $0$ & $1$ & $1$ & $(3)$  \\
			\hline
			$0$ & $1$ & $1$ & $1$ & $(2,3)$  \\
			\hline
		\end{tabular}\\


\end{tabular}
\end{table} 


\begin{table}[h!]
Процесс нахождения всех простых импликант системы частично определенных булевых функций проиллюстрирован таблицей:\ref{lb5}


\caption{\label{lb5}}
	\begin{tabular}{|c|c|c|c||c|}
		
		\hline
		$x_1$ & $x_2$ & $x_3$ & $x_4$ & признаки  \\
		\hline		
		
$0$ & $0$ & $0$ & $0$ & $2$ \\
\hline
$0$ & $0$ & $0$ & $0$ &  $3$ \\
\hline
$0$ & $0$ & $0$ & $0$ & $(2,3)$ \\
\hline
$0$ & $0$ & $0$ & $1$ & $(2)$ \\
\hline
$0$ & $1$ & $0$ & $0$ &  $(3)$ \\
\hline
$0$ & $1$ & $0$ & $1$ & $(3)$ \\
\hline
$0$ & $1$ & $1$ & $0$ & $(1)$  \\
\hline
$1$ & $0$ & $0$ & $0$ & $(2)$  \\
\hline
$1$ & $0$ & $0$ & $0$ &  $(3)$ \\
\hline
$1$ & $0$ & $0$ & $0$ &  $(2,3)$ \\
\hline
$1$ & $0$ & $0$ & $1$ & $(2)$  \\
\hline
$1$ & $0$ & $0$ & $1$ & $(3)$ \\
\hline
$1$ & $0$ & $0$ & $1$ & $(2,3)$ \\
\hline
$1$ & $0$ & $1$ & $0$ & $(1)$  \\
\hline
$1$ & $0$ & $1$ & $0$ &  $(2)$  \\
\hline
$1$ & $0$ & $1$ & $0$ & $(1,2)$  \\
\hline
$1$ & $0$ & $1$ & $1$ & $(1)$  \\
\hline
$1$ & $0$ & $1$ & $1$ &  $(2)$  \\
\hline
$1$ & $0$ & $1$ & $1$ & $(1,2)$  \\
\hline
$1$ & $1$ & $0$ & $1$ & $(1)$  \\
\hline
$1$ & $1$ & $0$ & $1$ &  $(3)$ \\
\hline
$1$ & $1$ & $0$ & $1$ & $(1,3)$ \\
\hline
$1$ & $1$ & $1$ & $0$ & $(1)$  \\
\hline
$1$ & $1$ & $1$ & $0$ &  $(3)$ \\
\hline
$1$ & $1$ & $1$ & $0$ & $(1,3)$ \\
\hline
$0$ & $0$ & $1$ & $0$ & $(1)$  \\
\hline
$0$ & $0$ & $1$ & $0$ & $(2)$ \\
\hline
$0$ & $0$ & $1$ & $0$ & $(1,2)$ \\
\hline
$0$ & $0$ & $1$ & $1$ & $(1)$  \\
\hline
$0$ & $0$ & $1$ & $1$ & $(2)$  \\
\hline
$0$ & $0$ & $1$ & $1$ & $(1,2)$  \\
\hline
$0$ & $1$ & $1$ & $1$ & $(1)$  \\
\hline

	\end{tabular}


\medskip

$\rightarrow$


\begin{tabular}{|c|c|c|c||c|}
	
	\hline
	$x_1$ & $x_2$ & $x_3$ & $x_4$ & признаки  \\
	\hline
	\hline
	$0$ & $1$ & $1$ & $1$ & $(1)$  \\
	\hline
	
\end{tabular}


\end{table} 









\end{document}
